%----------------------------------------------------------------------------------------
%	PACKAGES AND OTHER DOCUMENT CONFIGURATIONS
%----------------------------------------------------------------------------------------

\documentclass{article}

\usepackage{fancyhdr} % Required for custom headers
\usepackage{lastpage} % Required to determine the last page for the footer
\usepackage{extramarks} % Required for headers and footers
\usepackage{graphicx} % Required to insert images
\usepackage{lipsum} % Used for inserting dummy 'Lorem ipsum' text into the template
\usepackage{systeme}
\usepackage{amsfonts}
\usepackage{amsmath}
\usepackage{mathtools}
\usepackage{amssymb}
\usepackage{amsthm}

% Margins
\topmargin=-0.45in
\evensidemargin=0in
\oddsidemargin=0in
\textwidth=6.5in
\textheight=9.0in
\headsep=0.25in 

\linespread{1.1} % Line spacing

% Set up the header and footer
\pagestyle{fancy}
\lhead{\hmwkAuthorName} % Top left header
\chead{\hmwkClass\: } % Top center header
\rhead{\hmwkTitle} % Top right header
\lfoot{\lastxmark} % Bottom left footer
\cfoot{} % Bottom center footer
\rfoot{Page\ \thepage\ of\ \pageref{LastPage}} % Bottom right footer
\renewcommand\headrulewidth{0.4pt} % Size of the header rule
\renewcommand\footrulewidth{0.4pt} % Size of the footer rule

\setlength\parindent{0pt} % Removes all indentation from paragraphs

%----------------------------------------------------------------------------------------
%	DOCUMENT STRUCTURE COMMANDS
%	Skip this unless you know what you're doing
%----------------------------------------------------------------------------------------

% Header and footer for when a page split occurs within a problem environment
\newcommand{\enterProblemHeader}[1]{
\nobreak\extramarks{#1}{#1 continued on next page\ldots}\nobreak
\nobreak\extramarks{#1 (continued)}{#1 continued on next page\ldots}\nobreak
}

% Header and footer for when a page split occurs between problem environments
\newcommand{\exitProblemHeader}[1]{
\nobreak\extramarks{#1 (continued)}{#1 continued on next page\ldots}\nobreak
\nobreak\extramarks{#1}{}\nobreak
}

\setcounter{secnumdepth}{0} % Removes default section numbers
\newcounter{homeworkProblemCounter} % Creates a counter to keep track of the number of problems

\newcommand{\homeworkProblemName}{}
\newenvironment{homeworkProblem}[1][Problem \arabic{homeworkProblemCounter}]{ % Makes a new environment called homeworkProblem which takes 1 argument (custom name) but the default is "Problem #"
\stepcounter{homeworkProblemCounter} % Increase counter for number of problems
\renewcommand{\homeworkProblemName}{#1} % Assign \homeworkProblemName the name of the problem
\section{\homeworkProblemName} % Make a section in the document with the custom problem count
\enterProblemHeader{\homeworkProblemName} % Header and footer within the environment
}{
\exitProblemHeader{\homeworkProblemName} % Header and footer after the environment
}

\newcommand{\problemAnswer}[1]{ % Defines the problem answer command with the content as the only argument
\noindent\framebox[\columnwidth][c]{\begin{minipage}{0.98\columnwidth}#1\end{minipage}} % Makes the box around the problem answer and puts the content inside
}

\newcommand{\homeworkSectionName}{}
\newenvironment{homeworkSection}[1]{ % New environment for sections within homework problems, takes 1 argument - the name of the section
\renewcommand{\homeworkSectionName}{#1} % Assign \homeworkSectionName to the name of the section from the environment argument
\subsection{\homeworkSectionName} % Make a subsection with the custom name of the subsection
\enterProblemHeader{\homeworkProblemName\ [\homeworkSectionName]} % Header and footer within the environment
}{
\enterProblemHeader{\homeworkProblemName} % Header and footer after the environment
}
   
%----------------------------------------------------------------------------------------
%	NAME AND CLASS SECTION
%----------------------------------------------------------------------------------------

\newcommand{\hmwkTitle}{Analysis - August, 2015} % Assignment title
\newcommand{\hmwkDueDate}{Monday,\ January\ 1,\ 2012} % Due date
\newcommand{\hmwkClass}{UMD Math Qualifying Exam Solutions} % Course/class
\newcommand{\hmwkAuthorName}{Phil Wertheimer} % Your name

%----------------------------------------------------------------------------------------

\begin{document}

%----------------------------------------------------------------------------------------
%	PROBLEM 1
%----------------------------------------------------------------------------------------

\begin{homeworkProblem}
Let $m$ denote the Lebesgue measure restricted to the compact interval $[0,1]$. Let $\{f_{n}\}_{n\geq 1}$ be a sequence of continuously differentiable functions on $[0,1]$. Assume that:\\
\indent (a) $f_{n}(0) = 0$ for each $n \geq 1$,\\
\indent (b) $|f'_{n}| \leq x^{-1/2}$ a.e., \\
\indent (c) there exists a (Lebesgue) measurable function $h$ on $[0,1]$ such that $\lim_{n\to\infty}f_{n}'(x) = h(x)$ for each $x \in [0,1]$\\

Prove that there exists an absolutely continuous function $f$ such that $f_{n}$ converges to $f$ uniformly on $[0,1]$.\\

\problemAnswer{
By the fundamental theorem of calculus, for each $n\geq 1$ and each $x\in [0,1]$ we have 
$$f_{n}(x) = f_{n}(0) + \displaystyle\int_{0}^{x}f_{n}'(t)\ dt = \displaystyle\int_{0}^{x}f_{n}'(t)\ dt$$
Since $x^{-1/2} \in L^{1}[0,1]$, condition (b) with the Lebesgue Dominated Convergence Theorem gives
$$\lim_{n\to\infty}\displaystyle\int_{0}^{x}f_{n}'(t)\ dt = \displaystyle\int_{0}^{x}\lim_{n\to\infty}f_{n}'(t)\ dt = \displaystyle\int_{0}^{x}h(t)\ dt$$
Define $f(x) = \displaystyle\int_{0}^{x}h(t)\ dt$. Then $f$ is absolutely continuous since it is the integral of an $L^{1}$ function $h$, and we have $f_{n}\rightrightarrows f$ since $\forall x\in[0,1]$,
$$|f_{n}(x) - f(x)| = \left|\displaystyle\int_{0}^{x}f_{n}'(t)\ dt - \displaystyle\int_{0}^{x}h(t)\ dt\right| = \left|\displaystyle\int_{0}^{x}(f_{n}'(t) - h(t))\ dt\right| \leq \displaystyle\int_{0}^{x}|f_{n}'(t) - h(t)| \ dt \rightarrow 0$$
}
\end{homeworkProblem}

\clearpage

%----------------------------------------------------------------------------------------
%	PROBLEM 2
%----------------------------------------------------------------------------------------

\begin{homeworkProblem}

\begin{homeworkSection}{(a)} % Section within problem
Find an analytic isomorphism $f$ of the domain $U = \{z : |\Im z | < 1\}$ onto the unit disc \textbf{D} = $\{z : |z| < 1\}$, such that $f(0) = 0$, and evaluate $M = |f'(0)|$. \newline

\problemAnswer{
Define $g(z) = \displaystyle\frac{\pi}{2}z$; this maps $U$ into the horizontal strip $S := \{z : |\Im z | < \frac{\pi}{2}\}.$\\
Define $h(z) = e^{z}$; this maps $S$ into the right half plane.\\
Define $k(z) = \displaystyle\frac{z-1}{z+1}$; this maps the right half plane into \textbf{D}.\\
Now let $f(z) = (g\circ h\circ k)(z) = \displaystyle\frac{e^{\frac{\pi}{2}z} - 1}{e^{\frac{\pi}{2}z} + 1}$. This is an analytic isomorphism $U \rightarrow$ \textbf{D} with $f(0) = \displaystyle\frac{0}{2} = 0$. We have 
$$f'(z) = \displaystyle\frac{\frac{\pi}{2}(e^{\frac{\pi}{2}z} + 1) - (e^{\frac{\pi}{2}z} - 1)\frac{\pi}{2}e^{\frac{\pi}{2}z}}{(e^{\frac{\pi}{2}z} + 1)^{2}}$$
And so
$$M = |f'(0)| = \left|\displaystyle\frac{\frac{\pi}{2}(2) - 0}{4}\right| = \displaystyle\frac{\pi}{4}$$
}
\end{homeworkSection}

%--------------------------------------------

\begin{homeworkSection}{(b)} % Section within problem
Prove that if $g$ is any analytic isomorphism of \textbf{D} onto $U$ such that $g(0) = 0$, then $|g'(0)| \leq \displaystyle\frac{1}{M}$. \newline

\problemAnswer{
If $g: D \rightarrow U$ is an analytic isomorphism then $f\circ g: D \rightarrow D$ is also an analytic isomorphism and we have $(f\circ g)(0) = f(g(0)) = f(0) = 0$. Therefore by the Schwarz Lemma we have $|(f\circ g)'(0)| \leq 1$. By the chain rule, this implies that
$$|f'(g(0))\cdot g'(0)| \leq 1$$
Since $g(0) = 0$, this becomes
$$|f'(0) \cdot g'(0)| \leq 1 \Rightarrow |g'(0)| \leq \displaystyle\frac{1}{|f'(0)|} = \displaystyle\frac{1}{M}$$
}
\end{homeworkSection}

%--------------------------------------------

\end{homeworkProblem}

\clearpage

%----------------------------------------------------------------------------------------
%	PROBLEM 3
%----------------------------------------------------------------------------------------

\begin{homeworkProblem}
Let $m$ denote the Lebesgue measure on $\mathbb{R}$, and $E \subset \mathbb{R}$. Prove that $E$ is Lebesgue measurable if and only if there exist a $G_{\delta}$ set $G$, and a $F_{\sigma}$ set $F$ such that $F \subset E \subset G$ and $m(G\setminus F) = 0$. Conclude that in this case, $m(F) = m(G) = m(E)$. \newline

\problemAnswer{ 
First suppose $m^{*}(E) < \infty$. By definition of outer measure, for any $\epsilon > 0$, there is a collection $\{I_{k}\}$ of disjoint open intervals with $\sum_{k=1}^{\infty}\ell(I_{k}) < m^{*}(E) + \epsilon$. Defining $\mathcal{O} = \cup_{k=1}^{\infty}I_{k}$ gives an open set containing $E$ for which $m^{*}(O) < m^{*}(E) < \epsilon$. Since both have finite outer measure, we can apply excision and conclude that $m^{*}(\mathcal{O}\setminus E) < \epsilon$. \newline

If $m^{*}(E) = \infty$, define $E_{n} = E \cap [-n,n]$ and for each $n$ choose $\mathcal{O}_{n} \supset E_{n}$ with $m^{*}(\mathcal{O}_{n}\setminus E_{n}) < \frac{\epsilon}{2^{n}}$. Then $\mathcal{O} = \cup_{n=1}^{\infty}\mathcal{O}_{n}$ is an open set containing $E$, and by monotonicty and subadditivity of outer measure we have
$$m^{*}((\cup_{n=1}^{\infty}\mathcal{O}_{n})\setminus E) = m^{*}(\cup_{n=1}^{\infty}(\mathcal{O}_{n}\setminus E)) \leq m^{*}(\cup_{n=1}^{\infty}(\mathcal{O}_{n}\setminus E_{n})) \leq \sum_{n=1}^{\infty}m^{*}(\mathcal{O}_{n}\setminus E) < \sum_{n=1}^{\infty}\frac{\epsilon}{2^{n}} = \epsilon$$

We have thus shown that for any measurable set $E$, there exists an open set $\mathcal{O}\supset E$ with $m^{*}(\mathcal{O}\setminus E) < \epsilon$. Now, for each $n\geq 1$, choose an open set $\mathcal{O}_{n}$ containing $E$ for which $m^{*}(\mathcal{O}_{n} \setminus E) < \frac{1}{n}$. Define $G = \cap_{n=1}^{\infty}\mathcal{O}_{n}$. Then $G$ is by construction a $G_{\delta}$ set containing $E$. And by monotonicity, $m^{*}(G\setminus E) \leq m^{*}(\mathcal{O}_{n}\setminus E) < \frac{1}{n}$ for all $n$. Therefore $m^{*}(G\setminus E) = 0$.\newline

Now consider the measurable set $E^{c}$. By the above argument there exists a $G_{\delta}$ set $G \supset E^{c}$ with $m^{*}(G\setminus E) = 0$. But then $E \supset G^{c}$, where $G^{c}$ is an $F_{\sigma}$ set. And $m^{*}(E^{c}\setminus G^{c}) = m^{*}(E^{c}\cap G) = m^{*}(G\setminus E) = 0$.

When $E$ is measurable, by definition and monotonicity we have
$$m(G) = m^{*}(G) = m^{*}(G\cap E) + m^{*}(G\setminus E) = m^{*}(E) + 0 = m(E)$$
$$m(E) = m^{*}(E) = m^{*}(E\cap F) + m^{*}(E\setminus F) = m^{*}(F) + 0 = m(F) $$
Therefore $m(F) = m(G) = m(E)$.
}

\end{homeworkProblem}

\clearpage

%----------------------------------------------------------------------------------------
%	PROBLEM 4
%----------------------------------------------------------------------------------------

\begin{homeworkProblem}
Recall the circle mean value property for a function $u(x,y)$ defined in an open domain $\mathcal{D}$ :\\
If a circle $C$ of radius $r$ centerered at $z_{0} = (x_{0}, y_{0})$ is a boundary of a closed disc $B \subset \mathcal{D}$ then
$$u(x_{0},y_{0}) = \displaystyle\frac{1}{2\pi}\int_{C}u(z_{0} + re^{i\theta}) d\theta$$
Prove that a continous function $u(x,y)$ defined in an open domain $\mathcal{D}$ is harmonic if and only if it satisfies the circle mean value property.\\

\problemAnswer{ 
One direction is straightforward: suppose $u$ is harmonic in $\mathcal{D}$. Write $u = \Re(f(z))$ where $f$ is analytic in $\mathcal{D}$. By Cauchy's Integral Formula for analytic functions, we have
$$f(z_{0}) = \displaystyle\frac{1}{2\pi i}\displaystyle\int_{C}\displaystyle\frac{f(z)}{z-z_{0}}\ dz$$
Substituting $z=z_{0} + re^{i\theta}$ gives
$$ f(z_{0}) = \displaystyle\frac{1}{2\pi i}\displaystyle\int_{0}^{2\pi}\displaystyle\frac{f(z_{0} + re^{i\theta})}{re^{i}\theta}rie^{i\theta}\ d\theta = \displaystyle\frac{1}{2\pi}\displaystyle\int_{0}^{2\pi}f(z_{0} + re^{i\theta})\ d\theta $$

Taking the real part of both sides yields the circle mean value property.\newline

The converse is harder and requires some facts about harmonic functions. The first useful fact is that harmonic functions satisfy the Identity Principle: if $h$ is harmonic on a domain $D$ and $h \equiv 0$ on a non-empty open subset $U$ of $D$ then $h\equiv 0$ throughout $D$.\newline

\textit{Proof of Identity Princple for Harmonic Functions:} Since $h$ is harmonic, write $h = \Re(f)$ for $f$ holomorphic, say $f = h + ik$. Then the Cauchy-Riemann equations give
$$f' = h_{x} + ik_{x} = h_{y} - ih_{y}$$
Set $g = h_{x} - ih_{y}$. Then $g$ is continuous and satisfies the Cauchy-Riemann equations because $h_{xx} = -h_{yy}$ ($h$ is harmonic) and $h_{xy} = h_{yx}$. Therefore $g$ is holomorphic on $D$. Since $h = 0$ on $U$, $g=0$ on $U$. Hence the Identity Principle for holomorphic functions implies $g=0$ on $D$, and it follows that $h_{x} = h_{y} = 0$ on $D$. This implies $h$ is constant on $D$ and since it's zero in $U$ the constant must be $0$.\qed\newline

The second useful fact is that harmonic functions satisfy the Maximum Modulus Principle: if $h$ is a non-constant harmonic function in a region $\Omega \subset \mathbb{C}$ then $h$ has no maxima inside $\Omega$.\newline

\textit{Proof of MMP for Harmonic Functions:} Let $a \in \Omega$ and choose $\epsilon$ with $B(a, \epsilon) \subset \Omega$. There is a holomorphic $f$ on $B(a, \epsilon)$ with $h = \Re(f)$. $f$ is not constant on $B(a, \epsilon)$ - otherwise $h$ would also be constant there, and hence constant on $\Omega$ by the Identity Principle for Harmonic Functions. By the Open Mapping Theorem, $f(B(a, \epsilon))$ is an open set, so that $h$ takes values larger than $h(a)$ in $B(a, \epsilon)$.\qed\newline

Now, suppose $u(x,y)$ is continuous and satisfies the circle mean value property. Let $a \in \mathcal{D}$ and choose $\epsilon$ such that $\bar{B}(a, \epsilon) \subset \mathcal{D}$. Using the Poisson Kernel, we can construct a continuous function $h: \bar{B}(a, \epsilon) \rightarrow \mathbb{R}$ which is harmonic in $B(a, \epsilon)$ and $u(z) = h(z)$ on the boundary $\{|z-a| = \epsilon\}$. Since $u-h$ satisfies the mean value property and equals zero on the boundary, the Identity Principle for harmonic functions implies $u\equiv h$ in $B(a, \epsilon)$. In particular, $u$ is harmonic. Since this holds for any open ball in $\mathcal{D}$, $u$ is harmonic in $\mathcal{D}$.


}

\end{homeworkProblem}

\clearpage

%----------------------------------------------------------------------------------------
%	PROBLEM 5
%----------------------------------------------------------------------------------------

\begin{homeworkProblem}

Let $m$ denote the Lebesgue measure on $[0,1]$, and let $\{f_{n}\}_{n\geq 1} \subset L_{m}^{2}([0,1])$ be such that for each $\epsilon > 0$, there exists $N_{0} \geq 1$ so that 
$$\left|\left|\sup_{N > N_{0}}\left|\sum_{n=N_{0}}^{N}f_{n}\right|\right|\right|_{2} < \epsilon$$
Set $F_{n} = \sum_{k=1}^{n}f_{k}$, $F = \limsup\limits_{n}F_{n}$, and $G = \liminf\limits_{n}F_{n}$. Define
$$A= \{x \in [0,1] : F(x) > G(x)\}.$$

\begin{homeworkSection}{(a)}
Show that $A = \displaystyle\bigcup_{k=1}^{\infty}A_{k} = \displaystyle\bigcup_{k=1}^{\infty}\{x \in [0,1] : F(x) > 2^{-k} + G(x)\}$. \newline

\problemAnswer{
If $x\in A$, then $F(x) > G(x)$, so $F(x) - G(x) > 0$ and there must exist $k \in \mathbb{N}$ with $F(x) - G(x) > 2^{-k} > 0$. Thus $x \in A_{k}$, proving $A \subset \displaystyle\bigcup_{k=1}^{\infty}A_{k}$. Similarly, if $x \in \displaystyle\bigcup_{k=1}^{\infty}A_{k}$ then there exists $k\in\mathbb{N}$ with $F(x) > 2^{-k} + G(x) > G(x)$ so that $x\in A$, whence $\displaystyle\bigcup_{k=1}^{\infty}A_{k} \subset A$. Therefore the sets are equal.
}
\end{homeworkSection}

%--------------------------------------------

\begin{homeworkSection}{(b)}
Prove that for each $\epsilon > 0$ and $k \geq 1$, $m(A_{k}) \leq \epsilon^{2}2^{2k}$. Conclude that the series $\sum_{n=1}^{\infty}f_{n}$ converges a.e. \newline

\problemAnswer{
Let $\epsilon > 0$; choose $N_{0}$ such that
$$\left|\left|\sup_{N > N_{0}}\left|\sum_{n=N_{0}}^{N}f_{n}\right|\right|\right|_{2} < \displaystyle\frac{\epsilon^{2}}{4}$$

By Chebyshev's inequality,
\begin{align*}
    m(A_{k}) &= m(\{x\in[0,1] : F(x) - G(x) > 2^{-k}\} \\
    &\leq 2^{2k}\displaystyle\int|F(x)-G(x)|^{2} \\
    &= 2^{2k}\displaystyle\int|\limsup_{n}F_{n} - \liminf_{n}F_{n}|^{2}\\
    &\leq 2^{2k}\displaystyle\int\left|\sup_{n > N_{0}}F_{n}- \inf_{n > N_{0}}F_{n}\right|^{2} \\
    &= 2^{2k}\cdot 4 \displaystyle\int\left|\displaystyle\frac{\sup_{n > N_{0}}F_{n}- \inf_{n > N_{0}}F_{n}}{2}\right|^{2} \\
    &\leq 2^{2k}\cdot 4 \displaystyle\int|\sup_{n > N_{0}}|F_{n}-F_{N_{0}}||^{2}\\
    &< \epsilon^{2}2^{2k}
\end{align*}
Since $\epsilon$ was arbitrary, this shows $m(A_{k}) = 0$ for all $k$. Thus $m(A) = 0$, so that $F(x) = G(x)$ a.e. This means $\limsup_{n}\sum_{k=1}^{n}f_{k} = \liminf\sum_{k=1}^{n}f_{k}$ a.e. so that $\sum f_{n}$ converges a.e.
}
\end{homeworkSection}

%--------------------------------------------

\end{homeworkProblem}

\clearpage

%----------------------------------------------------------------------------------------
%	PROBLEM 6
%----------------------------------------------------------------------------------------

\begin{homeworkProblem}

\begin{homeworkSection}{(a)}
Find the domain of convergence of the power series
$$\displaystyle\sum_{n=1}^{\infty}\frac{n!(3n)!}{((2n)!)^{2}}z^{n}$$

\problemAnswer{ % Answer
By the ratio test,
$$\lim_{n\to\infty}\left|\displaystyle\frac{(n+1)!(3n+3)!}{((2n+2)!)^{2}} \displaystyle\frac{((2n)!)^{2}}{n!(3n)!}\right| = \lim_{n\to\infty}\left|\displaystyle\frac{(n+1)(3n+3)(3n+2)(3n+1)}{(2n+1)^{2}(2n+2)^{2}}\right| 
    = \lim_{n\to\infty}\left|\displaystyle\frac{27n^{4} + \hdots}{16n^{4} + \hdots}\right| 
    = \displaystyle\frac{27}{16}
$$

Therefore the radius of convergence is $\displaystyle\frac{16}{27}$, and so the domain of convergence is $\left\{z \in \mathbb{C} : |z| < \displaystyle\frac{16}{27}\right\}$.

}
\end{homeworkSection}

\begin{homeworkSection}{(b)} % Section within problem
Prove that the series diverges at every point on the boundary of the domain of convergence.\newline

\problemAnswer{
Each point on the boundary of the domain of convergence can be written as $z = \displaystyle\frac{16}{27}e^{i\theta}$. Recall Stirling's approximation:
$$n! \to \sqrt{2\pi n}\displaystyle\left(\frac{n}{e}\right)^{n}$$

Using this approximation, on the boundary the terms of the power series are given by
\begin{align*}
    a_{n} = \frac{n!(3n)!}{((2n)!)^{2}}z^{n} &\to
    \frac{\sqrt{2\pi n}\left(\displaystyle\frac{n}{e}\right)^{n}\sqrt{6\pi n}\left(\displaystyle\frac{3n}{e}\right)^{3n}}{\left(\sqrt{4\pi n}\left(\displaystyle\frac{2n}{e}\right)^{2n}\right)^{2}}\left(\frac{16}{27}e^{i\theta}\right)^{n} \\
    &= \displaystyle\frac{\displaystyle\frac{2\sqrt{3}\pi n\cdot n^{n}(3n)^{3n}}{e^{n}e^{3n}}}{\displaystyle\frac{4\pi n(2n)^{4n}}{e^{4n}}}\left(\frac{16}{27}e^{i\theta}\right)^{n} \\
    &= \displaystyle\frac{2\sqrt{3}\pi n(3n)^{3n}}{e^{4n}}\displaystyle\frac{e^{4n}}{4\pi n(2n)^{4n}}\left(\frac{16}{27}e^{i\theta}\right)^{n}
\end{align*}
Thus
\begin{align*}
    |a_{n}| \to \left|\displaystyle\frac{2\sqrt{3}\pi n\cdot n^{n}(3n)^{3n}}{4\pi n(2n)^{4n}}\left(\frac{16}{27}e^{i\theta}\right)^{n}\right| 
    &=
    \displaystyle\frac{\sqrt{3}}{2}\displaystyle\frac{n^{n}(3n)^{3n}}{(2n)^{4n}}\left(\frac{16}{27}\right)^{n} \\
    &= \displaystyle\frac{\sqrt{3}}{2}\displaystyle\frac{n^{n}3^{3n}n^{3n}}{2^{4n}n^{4n}}\left(\frac{2^{4}}{3^{3}}\right)^{n} \\
    &= \displaystyle\frac{\sqrt{3}}{2}
\end{align*}
Since the terms do not approach $0$, the series does not converge.
}

\end{homeworkSection}

%--------------------------------------------

\end{homeworkProblem}


\end{document}
